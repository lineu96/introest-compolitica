% Options for packages loaded elsewhere
\PassOptionsToPackage{unicode}{hyperref}
\PassOptionsToPackage{hyphens}{url}
%
\documentclass[
  ignorenonframetext,
  serif,
  professionalfont,
  usenames,
  dvipsnames,
  aspectratio = 169]{beamer}
\usepackage{pgfpages}
\setbeamertemplate{caption}[numbered]
\setbeamertemplate{caption label separator}{: }
\setbeamercolor{caption name}{fg=normal text.fg}
\beamertemplatenavigationsymbolsempty
% Prevent slide breaks in the middle of a paragraph
\widowpenalties 1 10000
\raggedbottom
\setbeamertemplate{part page}{
  \centering
  \begin{beamercolorbox}[sep=16pt,center]{part title}
    \usebeamerfont{part title}\insertpart\par
  \end{beamercolorbox}
}
\setbeamertemplate{section page}{
  \centering
  \begin{beamercolorbox}[sep=12pt,center]{section title}
    \usebeamerfont{section title}\insertsection\par
  \end{beamercolorbox}
}
\setbeamertemplate{subsection page}{
  \centering
  \begin{beamercolorbox}[sep=8pt,center]{subsection title}
    \usebeamerfont{subsection title}\insertsubsection\par
  \end{beamercolorbox}
}
\AtBeginPart{
  \frame{\partpage}
}
\AtBeginSection{
  \ifbibliography
  \else
    \frame{\sectionpage}
  \fi
}
\AtBeginSubsection{
  \frame{\subsectionpage}
}
\usepackage{amsmath,amssymb}
\usepackage{iftex}
\ifPDFTeX
  \usepackage[T1]{fontenc}
  \usepackage[utf8]{inputenc}
  \usepackage{textcomp} % provide euro and other symbols
\else % if luatex or xetex
  \usepackage{unicode-math} % this also loads fontspec
  \defaultfontfeatures{Scale=MatchLowercase}
  \defaultfontfeatures[\rmfamily]{Ligatures=TeX,Scale=1}
\fi
\usepackage{lmodern}
\ifPDFTeX\else
  % xetex/luatex font selection
\fi
% Use upquote if available, for straight quotes in verbatim environments
\IfFileExists{upquote.sty}{\usepackage{upquote}}{}
\IfFileExists{microtype.sty}{% use microtype if available
  \usepackage[]{microtype}
  \UseMicrotypeSet[protrusion]{basicmath} % disable protrusion for tt fonts
}{}
\makeatletter
\@ifundefined{KOMAClassName}{% if non-KOMA class
  \IfFileExists{parskip.sty}{%
    \usepackage{parskip}
  }{% else
    \setlength{\parindent}{0pt}
    \setlength{\parskip}{6pt plus 2pt minus 1pt}}
}{% if KOMA class
  \KOMAoptions{parskip=half}}
\makeatother
\usepackage{xcolor}
\newif\ifbibliography
\usepackage{color}
\usepackage{fancyvrb}
\newcommand{\VerbBar}{|}
\newcommand{\VERB}{\Verb[commandchars=\\\{\}]}
\DefineVerbatimEnvironment{Highlighting}{Verbatim}{commandchars=\\\{\}}
% Add ',fontsize=\small' for more characters per line
\newenvironment{Shaded}{}{}
\newcommand{\AlertTok}[1]{\textcolor[rgb]{1.00,0.00,0.00}{#1}}
\newcommand{\AnnotationTok}[1]{\textcolor[rgb]{0.00,0.50,0.00}{#1}}
\newcommand{\AttributeTok}[1]{#1}
\newcommand{\BaseNTok}[1]{#1}
\newcommand{\BuiltInTok}[1]{#1}
\newcommand{\CharTok}[1]{\textcolor[rgb]{0.00,0.50,0.50}{#1}}
\newcommand{\CommentTok}[1]{\textcolor[rgb]{0.00,0.50,0.00}{#1}}
\newcommand{\CommentVarTok}[1]{\textcolor[rgb]{0.00,0.50,0.00}{#1}}
\newcommand{\ConstantTok}[1]{#1}
\newcommand{\ControlFlowTok}[1]{\textcolor[rgb]{0.00,0.00,1.00}{#1}}
\newcommand{\DataTypeTok}[1]{#1}
\newcommand{\DecValTok}[1]{#1}
\newcommand{\DocumentationTok}[1]{\textcolor[rgb]{0.00,0.50,0.00}{#1}}
\newcommand{\ErrorTok}[1]{\textcolor[rgb]{1.00,0.00,0.00}{\textbf{#1}}}
\newcommand{\ExtensionTok}[1]{#1}
\newcommand{\FloatTok}[1]{#1}
\newcommand{\FunctionTok}[1]{#1}
\newcommand{\ImportTok}[1]{#1}
\newcommand{\InformationTok}[1]{\textcolor[rgb]{0.00,0.50,0.00}{#1}}
\newcommand{\KeywordTok}[1]{\textcolor[rgb]{0.00,0.00,1.00}{#1}}
\newcommand{\NormalTok}[1]{#1}
\newcommand{\OperatorTok}[1]{#1}
\newcommand{\OtherTok}[1]{\textcolor[rgb]{1.00,0.25,0.00}{#1}}
\newcommand{\PreprocessorTok}[1]{\textcolor[rgb]{1.00,0.25,0.00}{#1}}
\newcommand{\RegionMarkerTok}[1]{#1}
\newcommand{\SpecialCharTok}[1]{\textcolor[rgb]{0.00,0.50,0.50}{#1}}
\newcommand{\SpecialStringTok}[1]{\textcolor[rgb]{0.00,0.50,0.50}{#1}}
\newcommand{\StringTok}[1]{\textcolor[rgb]{0.00,0.50,0.50}{#1}}
\newcommand{\VariableTok}[1]{#1}
\newcommand{\VerbatimStringTok}[1]{\textcolor[rgb]{0.00,0.50,0.50}{#1}}
\newcommand{\WarningTok}[1]{\textcolor[rgb]{0.00,0.50,0.00}{\textbf{#1}}}
\usepackage{longtable,booktabs,array}
\usepackage{calc} % for calculating minipage widths
\usepackage{caption}
% Make caption package work with longtable
\makeatletter
\def\fnum@table{\tablename~\thetable}
\makeatother
\usepackage{graphicx}
\makeatletter
\newsavebox\pandoc@box
\newcommand*\pandocbounded[1]{% scales image to fit in text height/width
  \sbox\pandoc@box{#1}%
  \Gscale@div\@tempa{\textheight}{\dimexpr\ht\pandoc@box+\dp\pandoc@box\relax}%
  \Gscale@div\@tempb{\linewidth}{\wd\pandoc@box}%
  \ifdim\@tempb\p@<\@tempa\p@\let\@tempa\@tempb\fi% select the smaller of both
  \ifdim\@tempa\p@<\p@\scalebox{\@tempa}{\usebox\pandoc@box}%
  \else\usebox{\pandoc@box}%
  \fi%
}
% Set default figure placement to htbp
\def\fps@figure{htbp}
\makeatother
\setlength{\emergencystretch}{3em} % prevent overfull lines
\providecommand{\tightlist}{%
  \setlength{\itemsep}{0pt}\setlength{\parskip}{0pt}}
\setcounter{secnumdepth}{-\maxdimen} % remove section numbering
% Definição do esquema de cores:
% 1. UFPR - Azul com cinza.
% 2. DEST - Roxo com cinza.
% 3. LEG - Laranjado com cinza.
\def\mycolorscheme{1}

% Caminho para a imagem de fundo com aspecto 16x9.
% \def\pathtobg{config/ufpr-fachada-baixo-1.jpg}
% \def\pathtobg{config/ufpr-fundo.jpg}
% \def\pathtobg{config/ufpr-fundo.jpg}
\def\pathtobg{./config/ufpr-fundo-16x9.jpg}

% \providecommand{\tightlist}{%
%   \setlength{\itemsep}{0pt}\setlength{\parskip}{0pt}}
% ATTENTION: Redefine o comando acima que é definido pelo template.
% \renewcommand{\tightlist}{}
\renewcommand{\tightlist}{%
  \setlength{\itemsep}{0\baselineskip}
  \setlength{\parskip}{0.25\baselineskip}
}

% Logo na capa.
\titlegraphic{
  %\vspace{-1em}
  %\includegraphics[height=1.2cm]{config/dest-texto-2.png}\hspace{1em}
  %\includegraphics[height=1.8cm]{config/dsbd-logo-2x2.png}\hspace{1em}
  \includegraphics[height=1.8cm]{config/ufpr-transparent-600px.png}
}
%-----------------------------------------------------------------------

% Palladio.
% \usepackage[sc]{mathpazo}
% \linespread{1.05}         % Palladio needs more leading (space between lines)
% \usepackage[T1]{fontenc}

% Kurier.
% \usepackage[light, condensed, math]{kurier}
% \usepackage[T1]{fontenc}

% Iwona.
% \usepackage[math, light, condensed]{iwona}

% \usepackage{cmbright}
% \usepackage[charter]{mathdesign}
% \usepackage{palatino}

% Roboto (with Iwona for maths).
% \usepackage[math]{iwona}
% \usepackage[sfdefault, light, condensed]{roboto}

% Source Sans Pro (with Iwona for maths).
% \usepackage[math]{iwona}
% \usepackage[default, light]{sourcesanspro}

% Lato (with Iwona for maths).
% \usepackage[math]{iwona}
% \usepackage[default]{lato}

% Fira Sans (with Iwona for maths).
\usepackage[math, light]{iwona}
\usepackage[sfdefault,light]{FiraSans} %% option 'sfdefault' activates Fira Sans as the default text font
\usepackage[T1]{fontenc}
\renewcommand*\oldstylenums[1]{{\firaoldstyle #1}}

% Font for code. ----------------------------
% \usepackage[scaled=.75]{beramono}
\usepackage{inconsolata}

% ATTENTION: needs complile with xelatex: `$ xelatex file.tex`
% \usepackage{fontspec}
% \setmonofont{M+ 1m}
% \setmonofont{M+ 1mn}
% \setmonofont{M+ 2m}

%-----------------------------------------------------------------------

% \usepackage{lmodern}
\usepackage{amssymb, amsmath}
\usepackage[makeroom]{cancel}
% \usepackage{ifxetex, ifluatex}
\usepackage{fixltx2e} % provides \textsubscript
\usepackage[utf8]{inputenc}
\usepackage[shorthands=off,main=brazil]{babel}
\usepackage{graphicx}
\usepackage{xcolor}
\usepackage{setspace}
\usepackage{comment}
\usepackage{icomma}

%-----------------------------------------------------------------------
% Algumas configurações.

\setlength{\parindent}{0pt}
\setlength{\parskip}{6pt plus 2pt minus 1pt}
\setlength{\emergencystretch}{3em}  % prevent overfull lines
% \providecommand{\tightlist}{%
%   \setlength{\itemsep}{0pt}\setlength{\parskip}{0pt}}
\setcounter{secnumdepth}{0}

% Espaço vertical para o ambiente `quote`.
\let\oldquote\quote
\let\oldendquote\endquote
\renewenvironment{quote}{%
  \vspace{1em}\oldquote}{%
  \oldendquote\vspace{1em}}

%-----------------------------------------------------------------------
% Espaçamento entre items para itemize, enumerate e description.

% % itemize.
% \let\itemopen\itemize
% \let\itemclose\enditemize
% \renewenvironment{itemize}{%
%   \itemopen\addtolength{\itemsep}{0.25\baselineskip}}{\itemclose}
%
% % enumerate.
% \let\enumopen\enumerate
% \let\enumclose\endenumerate
% \renewenvironment{enumerate}{%
%   \enumopen\addtolength{\itemsep}{0.25\baselineskip}}{\enumclose}
%
% % description.
% \let\descopen\description
% \let\descclose\enddescription
% \renewenvironment{description}{%
%   \descopen\addtolength{\itemsep}{0.25\baselineskip}}{\descclose}

%-----------------------------------------------------------------------

% \usepackage[hang]{caption}
\usepackage{caption}
\captionsetup{font=footnotesize,
  labelfont={color=mycolor1, footnotesize},
  labelsep=period}

% \providecommand{\tightlist}{%
%   \setlength{\itemsep}{0pt}\setlength{\parskip}{0pt}}

%-----------------------------------------------------------------------

\usepackage{tikz}

% \def\pathtobg{/home/walmes/Projects/templates/COMMON/ufpr-fundo.jpg}
% \def\pathtobg{/home/walmes/Projects/templates/COMMON/ufpr-fundo-16x9.jpg}
% \def\pathtobg{/home/walmes/Projects/templates/COMMON/ufpr-fachada-dir-1.jpg}
% \def\pathtobg{/home/walmes/Projects/templates/COMMON/ufpr-fachada-esq-1.jpg}
% \def\pathtobg{/home/walmes/Projects/templates/COMMON/ufpr-perto-1.jpg}
% \def\pathtobg{/home/walmes/Projects/templates/COMMON/ufpr-fachada-baixo-1.jpg}

\ifx\pathtobg\undefined
\else
  \usebackgroundtemplate{
    \tikz[overlay, remember picture]
    \node[% opacity=0.3,
          at=(current page.south east),
          anchor=south east,
          inner sep=0pt] {
            \includegraphics[height=\paperheight, width=\paperwidth]{\pathtobg}};
  }
\fi

%-----------------------------------------------------------------------
% Definições de esquema de cores.

\ifx\mycolorscheme\undefined
  % UFPR.
  % http://www.color-hex.com/color-palette/2018
  \definecolor{mycolor1}{HTML}{015c93} % Título.
  \definecolor{mycolor2}{HTML}{363435} % Texto.
  \definecolor{mycolor3}{HTML}{015c93} % Estrutura.
  \definecolor{mycolor4}{HTML}{015c93} % Links.
  \definecolor{mycolor5}{HTML}{CECAC5} % Preenchimentos.
\else
  \if\mycolorscheme1
    % UFPR.
    \definecolor{mycolor1}{HTML}{015c93} % Título.
    \definecolor{mycolor2}{HTML}{363435} % Texto.
    \definecolor{mycolor3}{HTML}{015c93} % Estrutura.
    \definecolor{mycolor4}{HTML}{015c93} % Links.
    \definecolor{mycolor5}{HTML}{CECAC5} % Preenchimentos.
  \fi
  \if\mycolorscheme2
    % DEST.
    \definecolor{mycolor1}{HTML}{2a0e72} % Título.
    \definecolor{mycolor2}{HTML}{202E35} % Texto.
    \definecolor{mycolor3}{HTML}{2a0e72} % Estrutura.
    % \definecolor{mycolor3}{HTML}{8072a3} % Estrutura.
    \definecolor{mycolor4}{HTML}{2a0e72} % Links.
    % \definecolor{mycolor4}{HTML}{bfb9d1} % Links.
    % \definecolor{mycolor5}{HTML}{AEA79F} % Preenchimentos.
    \definecolor{mycolor5}{HTML}{CECAC5} % Preenchimentos.
  \fi
  \if\mycolorscheme3
    % LEG.
    \definecolor{mycolor2}{HTML}{363435} % Texto.
    % \definecolor{mycolor1}{HTML}{ff8000} % Título.
    % \definecolor{mycolor3}{HTML}{ff8000} % Estrutura.
    % \definecolor{mycolor4}{HTML}{ff8000} % Links.
    % \definecolor{mycolor1}{HTML}{E57300} % Título.
    % \definecolor{mycolor3}{HTML}{E57300} % Estrutura.
    % \definecolor{mycolor4}{HTML}{E57300} % Links.
    \definecolor{mycolor1}{HTML}{F67014} % Título.
    \definecolor{mycolor3}{HTML}{F67014} % Estrutura.
    \definecolor{mycolor4}{HTML}{F67014} % Links.
    % \definecolor{mycolor1}{HTML}{FE5C23} % Título.
    % \definecolor{mycolor3}{HTML}{FE5C23} % Estrutura.
    % \definecolor{mycolor4}{HTML}{FE5C23} % Links.
    \definecolor{mycolor5}{HTML}{222222} % Preenchimentos.
    \definecolor{mycolor5}{HTML}{383838} % Preenchimentos.
  \fi
\fi

\hypersetup{
  colorlinks=true,
  linkcolor=mycolor4,
  urlcolor=mycolor1,
  citecolor=mycolor1
}

%-----------------------------------------------------------------------
% ATTENTION: http://www.cpt.univ-mrs.fr/~masson/latex/Beamer-appearance-cheat-sheet.pdf

\usetheme{Boadilla}
\usecolortheme{default}

% \setbeamersize{text margin left=7mm, text margin right=7mm}
% \setbeamertemplate{frametitle}[default][left, leftskip=3mm]
% \addtobeamertemplate{frametitle}{\vspace{0.5em}}{}

\setbeamertemplate{caption}[numbered]
\setbeamertemplate{section in toc}[sections numbered]
\setbeamertemplate{subsection in toc}[subsections numbered]
\setbeamertemplate{sections/subsections in toc}[ball]{}
\setbeamertemplate{sections in toc}[ball]
\setbeamercolor{section number projected}{bg=mycolor1, fg=white}
\setbeamertemplate{blocks}[rounded]
\setbeamertemplate{navigation symbols}{}
\setbeamertemplate{frametitle continuation}{\gdef\beamer@frametitle{}}
% \setbeamertemplate{frametitle}[default][center]
% \setbeamertemplate{footline}[frame number]

\setbeamertemplate{enumerate items}[default]
\setbeamertemplate{itemize items}{\scriptsize\raise1.25pt\hbox{\donotcoloroutermaths$\blacktriangleright$}}

% Blocos.
% \addtobeamertemplate{block begin}{\vskip -\bigskipamount}{}
% \addtobeamertemplate{block end}{}{\vskip -\bigskipamount}
\addtobeamertemplate{block begin}{\vspace{0.5em}}{}
\addtobeamertemplate{block end}{}{\vspace{0.5em}}


% Rodapé.
\setbeamercolor{title in head/foot}{parent=subsection in head/foot}
\setbeamercolor{author in head/foot}{bg=mycolor4, fg=white}
\setbeamercolor{date in head/foot}{parent=subsection in head/foot, fg=mycolor3}

% Cabeçalho.
\setbeamercolor{section in head/foot}{bg=mycolor2, fg=mycolor4}
\setbeamercolor{subsection in head/foot}{bg=mycolor2, fg=white}

\setbeamercolor{title}{fg=mycolor1}       % Título dos slides.
\setbeamercolor{titlelike}{fg=title}
\setbeamercolor{subtitle}{fg=mycolor2}    % Subtítulo.
\setbeamercolor{institute in head/foot}{parent=palette primary} % Instituição.
\setbeamercolor{frametitle}{fg=mycolor1}  % De quadro.
\setbeamercolor{structure}{fg=mycolor3}   % Listas e rodapé.
\setbeamercolor{item projected}{bg=mycolor2}
\setbeamercolor{block title}{bg=mycolor5, fg=mycolor2}
\setbeamercolor{normal text}{fg=mycolor2} % Texto.
\setbeamercolor{caption name}{fg=normal text.fg}
% \setbeamercolor{footlinecolor}{fg=mycolor2, bg=mycolor5}
% \setbeamercolor{section in head/foot}{fg=mycolor2, bg=mycolor5}
\setbeamercolor{author in head/foot}{fg=white, bg=mycolor1}
\setbeamercolor{section in foot}{fg=mycolor4, bg=mycolor5}
\setbeamercolor{date in foot}{fg=mycolor4, bg=mycolor5}
\setbeamercolor{block title}{fg=white, bg=mycolor1}
\setbeamercolor{block body}{fg=black, bg=white!80!gray}
\setbeamercolor{block body}{fg=black, bg=white!80!gray}

% To remove empty brackets of \institution.
\makeatletter
\setbeamertemplate{footline}{
  \leavevmode%
  \hbox{%
    \begin{beamercolorbox}[
      wd=0.3\paperwidth, ht=2.25ex, dp=1ex, right]{author in head/foot}%
      \usebeamerfont{author in head/foot}\insertshortauthor{}\hspace*{1ex}
    \end{beamercolorbox}%
    \begin{beamercolorbox}[
      wd=0.6\paperwidth, ht=2.25ex, dp=1ex, left]{section in foot}%
      \usebeamerfont{title in head/foot}\hspace*{1ex}\insertshorttitle{}
      % \usebeamerfont{title in head/foot}\hspace*{1ex}\insertframetitle{}
    \end{beamercolorbox}%
    \begin{beamercolorbox}[
      wd=0.1\paperwidth, ht=2.25ex, dp=1ex, right]{date in foot}%
      \insertframenumber{}\hspace*{2ex}
    \end{beamercolorbox}
  }%
  \vskip0pt%
}
\makeatother

%-----------------------------------------------------------------------

% \usepackage{hyphenat}
\usepackage{changepage}

% Slide para o título das seções.
\AtBeginSection[]{
  \begin{frame}
    % \vfill
    \vspace{4cm}
    % \centering
    % \begin{beamercolorbox}[sep = 8pt, center, shadow = true, rounded = true]{title}
    \begin{beamercolorbox}{title}
      \begin{columns}
        \column{0.7\linewidth}
        {\LARGE\textbf \insertsectionhead}
      \end{columns}
    \end{beamercolorbox}
    \vfill
  \end{frame}
}

%-----------------------------------------------------------------------
%---- preamble-chunk.tex -----------------------------------------------

% Knitr.

% ATTENTION: this needs `\usepackage{xcolor}'.
\definecolor{color_line}{HTML}{333333}
\definecolor{color_back}{HTML}{DDDDDD}
% \definecolor{color_back}{HTML}{FF0000}

% ATTENTION: usa o fancyvrb.
% https://ctan.math.illinois.edu/macros/latex/contrib/fancyvrb/doc/fancyvrb-doc.pdf
% R input.
\usepackage{tcolorbox}
\ifcsmacro{Highlighting}{
  % Statment if it exists. ------------------
  \DefineVerbatimEnvironment{Highlighting}{Verbatim}{
    % frame=lines,     % Linha superior e inferior.
    % framerule=0.5pt, % Espessura da linha.
    framesep=2ex,    % Distância da linha para o texto.
    % rulecolor=\color{color_line},
    % numbers=right,
    fontsize=\footnotesize, % Tamanho da fonte.
    baselinestretch=0.8,    % Espaçamento entre linhas.
    commandchars=\\\{\}}
  % Margens do ambiente `Shaded'.
  % \fvset{listparameters={\setlength{\topsep}{-1em}}}
  % \renewenvironment{Shaded}{\vspace{-1ex}}{\vspace{-2ex}}
  \renewenvironment{Shaded}{
    \vspace{2pt}
    \begin{tcolorbox}[
      boxrule=0pt,      % Espessura do contorno.
      colframe=gray!10, % Cor do contorno.
      colback=gray!10,  % Cor de fundo da caixa.
      arc=1em,          % Raio para contornos arredondados.
      sharp corners,
      boxsep=0.5em,     % Margem interna.
      left=3pt, right=3pt, top=3pt, bottom=3pt, % Margens internas.
      grow to left by=0mm,
      grow to right by=6pt,
      ]
    }{
    \end{tcolorbox}
    \vspace{-3pt}
    }
  }{
  % Statment if it not exists. --------------
}

% R output e todo `verbatim'.
\makeatletter
\def\verbatim@font{\linespread{0.8}\ttfamily\footnotesize}
%\makeatother

% Cor de fundo e margens do `verbatim'.
\let\oldv\verbatim
\let\oldendv\endverbatim

\def\verbatim{%
  \par\setbox0\vbox\bgroup % Abre grupo.
  %\vspace{-5px}            % Reduz margem superior.
  \oldv                    % Chama abertura do verbatim.
}
\def\endverbatim{%
  \oldendv                 % Chama encerramento do verbatim.
  %\vspace{0cm}           % Controla margem inferior.
  \egroup%\fboxsep5px      % Fecha grupo.
  \noindent{{\usebox0}}\par
}

%-----------------------------------------------------------------------
%---- preamble-commands.tex --------------------------------------------

% Para fazer texto em duas colunas.
\newcommand{\mytwocolumns}[4]{
  % #1: Line width fraction for the left column , e.g. 0.5.
  % #2: Line width fraction for the right column.
  % #3: Content for the left column.
  % #4: Content for the right column.
  \begin{columns}[c]
    \begin{column}{#1\linewidth} %----------- left.
      #3
    \end{column} %--------------------------- left.
    \begin{column}{#2\linewidth} %----------- right.
      #4
    \end{column} %--------------------------- right.
  \end{columns}
}

%-----------------------------------------------------------------------
% Para fazer duas colunas no Rmd.

% Center vertical align.
\def\beginAHalfColumn{\begin{minipage}{0.49\textwidth}}%
\def\beginAlmostHalfColumn{\begin{minipage}{0.45\textwidth}}%
\def\beginAQuarterColumn{\begin{minipage}{0.23\textwidth}}%
\def\beginThreeQuartersColumn{\begin{minipage}{0.72\textwidth}}%
\def\beginAThirdColumn{\begin{minipage}{0.31\textwidth}}%
\def\beginTwoThirdsColumn{\begin{minipage}{0.64\textwidth}}%
\def\endColumns{\end{minipage}}%

% Top vertical align.
\def\beginAHalfColumnT{\begin{minipage}[t]{0.49\textwidth}}%
\def\beginAlmostHalfColumnT{\begin{minipage}[t]{0.45\textwidth}}%
\def\beginAQuarterColumnT{\begin{minipage}[t]{0.23\textwidth}}%
\def\beginThreeQuartersColumnT{\begin{minipage}[t]{0.72\textwidth}}%
\def\beginAThirdColumnT{\begin{minipage}[t]{0.31\textwidth}}%
\def\beginTwoThirdsColumnT{\begin{minipage}[t]{0.64\textwidth}}%

%---------------------------------------------------------------------
% Ambientes para frases como e sem imagem.

\newcommand{\myquote}[3]{
  % #1: caminho para a imagem.
  % #2: a frase/quotation.
  % #3: o autor.
  \begin{center}
    \begin{minipage}[c]{0.19\linewidth}
      \begin{center}
        \includegraphics[height=2.5cm]{#1}
      \end{center}
    \end{minipage}
    \begin{minipage}[c]{0.7\linewidth}
      \begin{flushright}
        \textit{#2}
        \vspace{1ex}

        -- #3
      \end{flushright}
    \end{minipage}
  \end{center}
}

\newcommand{\myphrase}[2]{
  % #1: a frase/quotation.
  % #2: o autor.
  \begin{center}
    \begin{minipage}[c]{0.19\linewidth}
    \end{minipage}
    \begin{minipage}[c]{0.7\linewidth}
      \begin{flushright}
        \textit{#1}
        \vspace{1ex}

        -- #2
      \end{flushright}
    \end{minipage}
  \end{center}
}

%-----------------------------------------------------------------------
% Comandos para texto em destaque.

% \newcommand{\hi}[1]{%
%   \textcolor{ubuntu_orange}{#1}\xspace
% }

\usepackage{xspace}

% URLs com letra miuda.
\newcommand{\myurl}[1]{%
  {\tiny \url{#1}}\xspace
}

% Botões.
\newcommand{\btn}[1]{%
  \beamergotobutton{#1}\xspace
}

% Texto grande centralizado.
\newcommand{\centertitle}[1]{%
  \begin{center}
    {\LARGE \bfseries \hi{#1}}
  \end{center}
}

%-----------------------------------------------------------------------
\usepackage{bookmark}
\IfFileExists{xurl.sty}{\usepackage{xurl}}{} % add URL line breaks if available
\urlstyle{same}
\hypersetup{
  pdfauthor={Prof.~Me. Lineu Alberto Cavazani de Freitas },
  hidelinks,
  pdfcreator={LaTeX via pandoc}}

\title{\hfill\break
\textbf{Introdução ao R}}
\author{Prof.~Me. Lineu Alberto Cavazani de Freitas \vspace{-0.5cm}}
\date{}

\begin{document}
\frame{\titlepage}

\begin{frame}{Estatística e o desenvolvimento computacional}
\phantomsection\label{estatuxedstica-e-o-desenvolvimento-computacional}
\beginAHalfColumn

\begin{itemize}
\tightlist
\item
  A popularização da Estatística se dá graças ao
  \textbf{desenvolvimento computacional}.
\end{itemize}

\vspace{0.3cm}

\begin{itemize}
\tightlist
\item
  Os computadores pessoais tornaram os métodos estatísticos mais
  acessíveis ao público geral por meio de \textbf{softwares} que
  implementam as metodologias.
\end{itemize}

\vspace{0.3cm}

\begin{itemize}
\tightlist
\item
  Devido ao avanço computacional, houve um aumento considerável na
  capacidade de \textbf{produzir e armazenar dados} provenientes das
  mais diversas fontes.
\end{itemize}

\endColumns
\beginAHalfColumn

\begin{figure}

{\centering \includegraphics[width=0.6\linewidth]{./img/desenvolvimento-computacional} 

}

\caption{Extraído de \href{https://cdn.pixabay.com/photo/2020/04/04/04/23/graph-5000784_1280.png}{pixabay.com.}}\label{fig:unnamed-chunk-2}
\end{figure}

\endColumns
\end{frame}

\begin{frame}{Estatística e o desenvolvimento computacional}
\phantomsection\label{estatuxedstica-e-o-desenvolvimento-computacional-1}
\beginAHalfColumn

\begin{itemize}
\tightlist
\item
  Graças ao avanço computacional podemos lidar com a manipulação de
  \textbf{grandes conjuntos de dados}.
\end{itemize}

\vspace{0.3cm}

\begin{itemize}
\tightlist
\item
  Este grande volume de dados também força o
  \textbf{desenvolvimento dos métodos estatísticos} e softwares para
  análise de dados.
\end{itemize}

\vspace{0.3cm}

\begin{itemize}
\tightlist
\item
  A capacidade computacional atual também desperta o interesse por
  \textbf{métodos estatísticos computacionalmente intensivos}.
\end{itemize}

\endColumns
\beginAHalfColumn

\begin{figure}

{\centering \includegraphics[width=0.6\linewidth]{./img/ti} 

}

\caption{Extraído de \href{https://cdn.pixabay.com/photo/2015/04/14/23/17/it-business-722950_1280.png}{pixabay.com.}}\label{fig:unnamed-chunk-3}
\end{figure}

\endColumns
\end{frame}

\begin{frame}{Ferramentas para análises estatísticas}
\phantomsection\label{ferramentas-para-anuxe1lises-estatuxedsticas}
Existem diversas ferramentas disponíveis:

\beginAHalfColumn

\begin{itemize}
\item
  R;
\item
  Python;
\item
  SAS;
\item
  Spss;
\item
  Biostat;
\item
  Minitab;
\item
  Tableau;
\item
  Stata;
\item
  E diversas outras.
\end{itemize}

\endColumns
\beginAHalfColumn

\begin{figure}

{\centering \includegraphics[width=0.8\linewidth]{./img/programacao} 

}

\caption{Extraído de \href{https://cdn.pixabay.com/photo/2018/06/08/00/48/developer-3461405_960_720.png}{pixabay.com.}}\label{fig:unnamed-chunk-4}
\end{figure}

\endColumns
\end{frame}

\begin{frame}{Ferramentas para análises estatísticas}
\phantomsection\label{ferramentas-para-anuxe1lises-estatuxedsticas-1}
Existem diversas ferramentas disponíveis:

\beginAHalfColumn

\begin{itemize}
\item
  \textbf{R};
\item
  Python;
\item
  SAS;
\item
  Spss;
\item
  Biostat;
\item
  Minitab;
\item
  Tableau;
\item
  Stata;
\item
  E diversas outras.
\end{itemize}

\endColumns
\beginAHalfColumn

\begin{figure}

{\centering \includegraphics[width=0.8\linewidth]{./img/rlogo} 

}

\caption{Logo do R.}\label{fig:unnamed-chunk-5}
\end{figure}

\endColumns
\end{frame}

\begin{frame}{R}
\phantomsection\label{r}
\begin{itemize}
\tightlist
\item
  R é uma linguagem e ambiente para \textbf{computação estatística} e
  \textbf{gráficos}.
\end{itemize}

\vspace{0.3cm}

\begin{itemize}
\tightlist
\item
  É \textbf{livre} e de \textbf{código aberto}.

  \begin{itemize}
  \tightlist
  \item
    Livre (free): usuários tem liberdade de:

    \begin{enumerate}
    \tightlist
    \item
      executar como desejar e para qualquer propósito.
    \item
      estudar o funcionamento e adapta-lo à necessidades específicas.
    \item
      distribuir cópias de versões originais e modificadas.
    \end{enumerate}
  \item
    Código aberto (open source): o acesso ao código fonte é gratuito.
  \end{itemize}
\end{itemize}
\end{frame}

\begin{frame}{R}
\phantomsection\label{r-1}
\begin{itemize}
\tightlist
\item
  Muito popular no meio \textbf{acadêmico} e tem uso cada vez maior no
  meio \textbf{corporativo}.

  \begin{itemize}
  \tightlist
  \item
    É acessível e gratuito.
  \item
    Tem diversas técnicas e aplicações possíveis.
  \end{itemize}
\end{itemize}

\vspace{0.3cm}

\begin{itemize}
\tightlist
\item
  Tem potencial uso em todas as etapas do processo de análise de dados.

  \begin{itemize}
  \tightlist
  \item
    Obtenção, importação, manipulação e tratamento.
  \item
    Análise exploratória.
  \item
    Ajuste de modelos estatísticos, modelos de aprendizado de máquina,
    dentre outros.
  \item
    Elaboração de relatórios dinâmicos e reproduzíveis.
  \end{itemize}
\end{itemize}
\end{frame}

\begin{frame}{O que é o R}
\phantomsection\label{o-que-uxe9-o-r}
\begin{itemize}
\item
  O R é uma \textbf{linguagem de programação}.
\item
  Uma linguagem de programação é a forma que nós nos comunicamos com o
  computador.
\item
  Trabalharemos com a ideia de \textbf{interface de linha de comando}
  (command line interface): escreveremos códigos para que o computador
  entenda e execute a tarefa de interesse.
\item
  Faremos a análise de dados escrevendo \textbf{funções e scripts}, não
  apontando, clicando e arrastando caixas.
\item
  Para quem nunca programou, parece assustador. Mas o R é fácil de
  aprender e guiado a análise de dados.
\end{itemize}
\end{frame}

\begin{frame}{O que é o R}
\phantomsection\label{o-que-uxe9-o-r-1}
\begin{itemize}
\tightlist
\item
  É possível instalar e usar o R nos principais sistemas operacionais.
\end{itemize}

\vspace{0.3cm}

\begin{itemize}
\tightlist
\item
  Assim como vários outros softwares livres e de código aberto, o R tem
  lançamentos frequentes de \textbf{versões}.
\end{itemize}

\vspace{0.3cm}

\begin{itemize}
\tightlist
\item
  A comunidade R é altamente ativa com usuários no mundo todo que
  contribuem, desenvolvem pacotes e ajudam uns aos outros por meio de
  materiais online como listas de discussão e tutoriais.
\end{itemize}
\end{frame}

\begin{frame}{R e os pacotes}
\phantomsection\label{r-e-os-pacotes}
\beginAHalfColumn

\begin{itemize}
\tightlist
\item
  \textbf{Pacotes R} são coleções de funções R, dados e código
  compilado.
\end{itemize}

\vspace{0.3cm}

\begin{itemize}
\tightlist
\item
  O R já vem com um conjunto de pacotes por padrão e outros podem ser
  adicionados para estender os recursos.
\end{itemize}

\vspace{0.3cm}

\begin{itemize}
\tightlist
\item
  Os pacotes hoje disponíveis são o resultado de anos de colaboração de
  pessoas de todo o mundo.
\end{itemize}

\vspace{0.3cm}

\begin{itemize}
\tightlist
\item
  Uma das funções do CRAN é hospedar diversos pacotes complementares.
\end{itemize}

\endColumns
\beginAHalfColumn

\begin{figure}

{\centering \includegraphics[width=0.6\linewidth]{./img/pacote} 

}

\caption{Extraído de \href{https://cdn.pixabay.com/photo/2014/12/21/23/35/parcel-575623_960_720.png}{pixabay.com.}}\label{fig:unnamed-chunk-6}
\end{figure}

\endColumns
\end{frame}

\begin{frame}[fragile]{R-base}
\phantomsection\label{r-base}
\begin{itemize}
\tightlist
\item
  O R ``base'' contém 15 pacotes básicos necessários para executar o R e
  as funções mais fundamentais.
\end{itemize}

\vspace{0.2cm}

\begin{itemize}
\tightlist
\item
  As funções já vem disponíveis, prontas para chamada e uso.
\end{itemize}

\vspace{0.2cm}

\beginAHalfColumn

\begin{enumerate}
\tightlist
\item
  \texttt{base}
\item
  \texttt{compiler}
\item
  \texttt{datasets}
\item
  \texttt{grDevices}
\item
  \texttt{graphics}
\item
  \texttt{grid}
\item
  \texttt{methods}
\item
  \texttt{parallel}
\end{enumerate}

\endColumns
\beginAHalfColumn

\begin{enumerate}
\setcounter{enumi}{8}
\tightlist
\item
  \texttt{splines}
\item
  \texttt{stats}
\item
  \texttt{stats4}
\item
  \texttt{tcltk}
\item
  \texttt{tools}
\item
  \texttt{translations}
\item
  \texttt{utils}.
\end{enumerate}

\endColumns
\end{frame}

\begin{frame}[fragile]{R-recommended}
\phantomsection\label{r-recommended}
\begin{itemize}
\tightlist
\item
  O R vem equipado com outros 15 pacotes ``recomendados''.
\end{itemize}

\vspace{0.2cm}

\begin{itemize}
\tightlist
\item
  Apesar de já instaladas, estas bibliotecas precisam ser chamadas para
  que seja possível usar as funções.
\end{itemize}

\vspace{0.2cm}

\beginAHalfColumn

\begin{enumerate}
\tightlist
\item
  \texttt{KernSmooth}
\item
  \texttt{MASS}
\item
  \texttt{Matrix}
\item
  \texttt{boot}
\item
  \texttt{class}
\item
  \texttt{cluster}
\item
  \texttt{codetools}
\item
  \texttt{foreign}
\end{enumerate}

\endColumns
\beginAHalfColumn

\begin{enumerate}
\setcounter{enumi}{8}
\tightlist
\item
  \texttt{lattice}
\item
  \texttt{mgcv}
\item
  \texttt{nlme}
\item
  \texttt{nnet}
\item
  \texttt{rpart}
\item
  \texttt{spatial}
\item
  \texttt{survival}.
\end{enumerate}

\endColumns
\end{frame}

\begin{frame}{Outros pacotes}
\phantomsection\label{outros-pacotes}
\beginAHalfColumn

\begin{itemize}
\tightlist
\item
  Você pode facilmente obter e instalar pacotes além dos 30 que já vem
  com a instalação tradicional do R.
\end{itemize}

\vspace{0.3cm}

\begin{itemize}
\tightlist
\item
  A principal fonte de pacotes é o próprio CRAN, que hoje conta com
  \textbf{mais de 19000 pacotes}.
\end{itemize}

\vspace{0.3cm}

\begin{itemize}
\tightlist
\item
  Fontes secundárias envolvem páginas web e repositórios como github,
  onde desenvolvedores mantém pacotes em desenvolvimento.
\end{itemize}

\endColumns
\beginAHalfColumn

\begin{figure}

{\centering \includegraphics[width=0.9\linewidth]{./img/list-packages} 

}

\caption{Lista de pacotes disponíveis por nome no CRAN.}\label{fig:unnamed-chunk-7}
\end{figure}

\endColumns
\end{frame}

\begin{frame}{Em resumo}
\phantomsection\label{em-resumo}
O R fornece

\begin{itemize}
\tightlist
\item
  Diversos recursos de Estatística.
\item
  Diversos recursos gráficos.
\item
  Uma vasta coleção de pacotes oficiais e não oficiais.
\item
  Uma linguagem de programação bem desenvolvida, simples e eficaz.
\item
  Possibilidade de instalação e uso nos mais comuns sistemas
  operacionais.
\item
  Possibilidade de uso de códigos C, C++ e Fortran para tarefas
  computacionalmente intesivas.
\item
  Documentação padronizada.
\end{itemize}
\end{frame}

\begin{frame}{IDEs e Editores}
\phantomsection\label{ides-e-editores}
\beginAHalfColumn

\begin{itemize}
\tightlist
\item
  Existem softwares adicionais úteis para ajudar a programar de forma
  mais rápida e eficiente.
\end{itemize}

\vspace{0.3cm}

\begin{itemize}
\tightlist
\item
  As IDE's (\emph{Integrated Development Environment}) são softwares que
  oferecem algumas facilidades para se programar em determinada
  linguagem.
\end{itemize}

\vspace{0.3cm}

\begin{itemize}
\tightlist
\item
  Já os editores tendem a ser úteis para múltiplas linguagens e fornecem
  mais alternativas de customização.
\end{itemize}

\endColumns
\beginAHalfColumn

\begin{itemize}
\tightlist
\item
  Para trabalhar em R, dentre IDEs e editores, o RStudio IDE é a opção
  mais famosa.
\end{itemize}

\vspace{1cm}

\begin{figure}

{\centering \includegraphics[width=0.6\linewidth]{./img/rstudiologo} 

}

\caption{Logo do RStudio.}\label{fig:unnamed-chunk-8}
\end{figure}

\endColumns
\end{frame}

\begin{frame}{Rstudio}
\phantomsection\label{rstudio}
\beginAHalfColumn

\begin{itemize}
\tightlist
\item
  Atenção:

  \begin{itemize}
  \tightlist
  \item
    O R é a \textbf{linguagem}.
  \item
    O RStudio é a \textbf{interface}.
  \item
    Quem faz o trabalho (manda a solicitação para que o computador
    execute) é o R!
  \end{itemize}
\end{itemize}

\endColumns
\beginAHalfColumn

\begin{figure}

{\centering \includegraphics[width=0.6\linewidth]{./img/atencao} 

}

\caption{Extraído de \href{https://cdn.pixabay.com/photo/2013/04/01/10/57/exclamation-mark-98739_1280.png}{pixabay.com.}}\label{fig:unnamed-chunk-9}
\end{figure}

\endColumns
\end{frame}

\section{Abrindo o RStudio}\label{abrindo-o-rstudio}

\begin{frame}{Abrindo o RStudio}
\phantomsection\label{abrindo-o-rstudio-1}
\beginAHalfColumn

\begin{itemize}
\tightlist
\item
  O RStudio como IDE tem o papel de facilitar o trabalho em R.
\end{itemize}

\vspace{0.3cm}

\begin{itemize}
\tightlist
\item
  Portanto, ao abrir o RStudio, tudo que você precisa deve estar à
  mostra e com fácil acesso.
\end{itemize}

\vspace{0.3cm}

\begin{itemize}
\tightlist
\item
  Ao abrir o RStudio você verá uma estrutura organizada em paineis.
\end{itemize}

\endColumns
\beginAHalfColumn

\begin{figure}

{\centering \includegraphics[width=0.6\linewidth]{./img/rstudiologo} 

}

\caption{Logo do RStudio.}\label{fig:unnamed-chunk-10}
\end{figure}

\endColumns
\end{frame}

\begin{frame}{Abrindo o RStudio}
\phantomsection\label{abrindo-o-rstudio-2}
\begin{figure}

{\centering \includegraphics[width=0.85\linewidth]{./img/rstudio} 

}

\caption{Tela inicial do RStudio.}\label{fig:unnamed-chunk-11}
\end{figure}
\end{frame}

\begin{frame}[fragile]{Paineis}
\phantomsection\label{paineis}
\begin{itemize}
\item
  No \textbf{canto superior esquerdo} é onde digitamos o código R que
  será executado: o \textbf{editor}.

  \begin{itemize}
  \tightlist
  \item
    Se o R está recém instalado ou a sessão é nova, crie um arquivo .R
    clicando em
    \texttt{File\ \textgreater{}\ New\ File\ \textgreater{}\ R} ou use
    as teclas de atalho \texttt{Ctrl\ +\ Shift\ +\ N}.
  \end{itemize}
\item
  No \textbf{canto inferior esquerdo} é onde o terminal R está, é ali
  onde o código é interpretado e executado: o \textbf{console}.
\item
  No canto superior direito são mostradas informações do ambiente de
  trabalho, histórico, conexões, etc.
\item
  No canto inferior direito são apresentados os arquivos da pasta de
  trabalho, gráficos, pacotes, documentação etc.
\end{itemize}
\end{frame}

\begin{frame}{Editor}
\phantomsection\label{editor}
\begin{figure}

{\centering \includegraphics[width=0.85\linewidth]{./img/editor} 

}

\caption{Tela inicial do RStudio. Foco no editor.}\label{fig:unnamed-chunk-12}
\end{figure}
\end{frame}

\begin{frame}{Console}
\phantomsection\label{console}
\begin{figure}

{\centering \includegraphics[width=0.85\linewidth]{./img/console} 

}

\caption{Tela inicial do RStudio. Foco no console.}\label{fig:unnamed-chunk-13}
\end{figure}
\end{frame}

\begin{frame}{Ambiente, histórico, conexões}
\phantomsection\label{ambiente-histuxf3rico-conexuxf5es}
\begin{figure}

{\centering \includegraphics[width=0.85\linewidth]{./img/env} 

}

\caption{Tela inicial do RStudio. Foco no ambiente, histórico e conexões.}\label{fig:unnamed-chunk-14}
\end{figure}
\end{frame}

\begin{frame}{Arquivos, gráficos, pacotes e documentação}
\phantomsection\label{arquivos-gruxe1ficos-pacotes-e-documentauxe7uxe3o}
\begin{figure}

{\centering \includegraphics[width=0.85\linewidth]{./img/files} 

}

\caption{Tela inicial do RStudio. Foco nos arquivos, pacotes e documentação.}\label{fig:unnamed-chunk-15}
\end{figure}
\end{frame}

\begin{frame}[fragile]{Principais elementos}
\phantomsection\label{principais-elementos}
Em resumo:

\begin{itemize}
\tightlist
\item
  \textbf{Editor}: onde escrevemos os códigos.
\item
  \textbf{Console}: onde os resultados são printados.
\item
  \textbf{Environment}: mostra todos os objetos criados.
\item
  \textbf{History}: mostra todos os códigos executados.
\item
  \textbf{Files}: mostra os arquivos no diretório atual.
\item
  \textbf{Plots}: mostra os outputs de códigos que geram gráficos.
\item
  \textbf{Packages}: mostra os pacotes instalados.
\item
  \textbf{Help}: mostra a documentação de funções e pacotes.
\end{itemize}

\textbf{Dica}: Para obter todos os atalhos da interface digite pressione
\texttt{ALT+SHIFT+K}.
\end{frame}

\begin{frame}[fragile]{Customizando o RStudio}
\phantomsection\label{customizando-o-rstudio}
\beginAHalfColumn

\begin{itemize}
\tightlist
\item
  O RStudio permite customizar os elementos.
\end{itemize}

\vspace{0.3cm}

\begin{itemize}
\tightlist
\item
  Experimente acessar
  \texttt{Tools\ \textgreater{}\ Global\ Options\ \textgreater{}\ Appearance}
  para trocar temas, ordem dos paineis, tamanho da fonte, etc.
\end{itemize}

\vspace{0.3cm}

\begin{itemize}
\tightlist
\item
  Deixe o RStudio confortável para você.
\end{itemize}

\endColumns
\beginAHalfColumn

\begin{figure}

{\centering \includegraphics[width=0.85\linewidth]{./img/aparencia} 

}

\caption{Tela inicial do RStudio.}\label{fig:unnamed-chunk-16}
\end{figure}

\endColumns
\end{frame}

\section{Trabalhando com R no
RStudio}\label{trabalhando-com-r-no-rstudio}

\begin{frame}[fragile]{Primeiros passos}
\phantomsection\label{primeiros-passos}
\begin{enumerate}
\item
  Crie uma pasta em algum lugar do seu computador para trabalhar.
\item
  Abra o RStudio e defina o diretório de trabalho.

  \begin{itemize}
  \tightlist
  \item
    O diretório de trabalho é a pasta do R onde estamos trabalhando.
  \item
    \texttt{Session\ \textgreater{}\ Set\ Working\ Directory\ \textgreater{}\ Choose\ Directory...}
    ou \texttt{CTRL\ +\ SHIFT\ +\ H}.

    \begin{itemize}
    \tightlist
    \item
      Ambas as opções darão a possibilidade de escolher a pasta de
      interesse no computador onde manteremos nossos arquivos.
    \end{itemize}
  \end{itemize}
\end{enumerate}
\end{frame}

\begin{frame}[fragile]{Primeiros passos}
\phantomsection\label{primeiros-passos-1}
\begin{enumerate}
\setcounter{enumi}{2}
\tightlist
\item
  Crie um arquivo com extensão .R.

  \begin{itemize}
  \tightlist
  \item
    \texttt{File\ \textgreater{}\ New\ File\ \textgreater{}\ R\ script}.
  \item
    \texttt{CTRL\ +\ SHIFT\ +\ N}.
  \end{itemize}
\item
  Salve seu script.

  \begin{itemize}
  \tightlist
  \item
    Esta etapa funciona como qualquer arquivo do computador.
  \item
    \texttt{File\ \textgreater{}\ Save} ou \texttt{CTRL\ +\ S}.
  \item
    De um nome para seu arquivo.

    \begin{itemize}
    \tightlist
    \item
      Opte por nomes curtos, intuitivos e evite espaços, acentos e
      caracteres.
    \end{itemize}
  \end{itemize}
\end{enumerate}

Agora estamos prontos para trabalhar!
\end{frame}

\begin{frame}[fragile]{Instruções e comentários}
\phantomsection\label{instruuxe7uxf5es-e-comentuxe1rios}
\begin{itemize}
\tightlist
\item
  Uma \textbf{instrução} é um código a ser executado.
\end{itemize}

\vspace{0.3cm}

\begin{itemize}
\tightlist
\item
  Um \textbf{comentário} é algo que escrevemos no script mas que não
  temos interesse que seja executado.
\end{itemize}

\vspace{0.3cm}

\begin{itemize}
\tightlist
\item
  Comentários podem e devem ser usados para \textbf{documentar} o
  código.
\end{itemize}

\vspace{0.3cm}

\begin{itemize}
\tightlist
\item
  Tudo que vier após \texttt{\#} é um comentário e não será executado
  pela linguagem.
\end{itemize}
\end{frame}

\begin{frame}[fragile]{Executando}
\phantomsection\label{executando}
\begin{itemize}
\tightlist
\item
  Para executar uma instrução no RStudio basta ir até a linha de
  interesse e teclar \texttt{CTRL\ +\ ENTER}.
\end{itemize}

\vspace{0.3cm}

\begin{itemize}
\tightlist
\item
  O código é interpretado, executado e o resultado é mostrado na tela.
\end{itemize}

\vspace{0.3cm}

\begin{itemize}
\tightlist
\item
  Uma recomendação para scripts mais organizados é não ultrapassar
  \(72\) ou \(80\) caracteres por linha.
\end{itemize}

\vspace{0.3cm}

\begin{itemize}
\tightlist
\item
  Outra recomendação diz respeito à indentação ou alinhamento do código.

  \begin{itemize}
  \tightlist
  \item
    Use \texttt{CTRL+i} no RStudio para indentar automaticamente.
  \end{itemize}
\end{itemize}
\end{frame}

\section{R como calculadora}\label{r-como-calculadora}

\begin{frame}{R como calculadora}
\phantomsection\label{r-como-calculadora-1}
\begin{itemize}
\item
  O R pode ser usado como uma poderosa \textbf{calculadora científica}.
\item
  Os operadores seguem uma hierarquia, ou seja, uma ordem de
  precedência.

  \begin{itemize}
  \tightlist
  \item
    Inicialmente são efetuadas as operações entre parênteses seguindo a
    ordem: exponenciação, multiplicação/divisão e por fim
    adição/subtração.
  \end{itemize}
\item
  Para utilizar os operadores no R basta digitar os valores e a operação
  diretamente no console (caso queira ver somente o resultado) ou no
  editor (caso deseje salvar o código no arquivo .R).
\end{itemize}
\end{frame}

\begin{frame}[fragile]{Operações aritméticas básicas}
\phantomsection\label{operauxe7uxf5es-aritmuxe9ticas-buxe1sicas}
\beginAThirdColumn

\begin{Shaded}
\begin{Highlighting}[]
\CommentTok{\# Soma}
\DecValTok{1} \SpecialCharTok{+} \DecValTok{1} 
\end{Highlighting}
\end{Shaded}

\begin{verbatim}
## [1] 2
\end{verbatim}

\begin{Shaded}
\begin{Highlighting}[]
\CommentTok{\# Subtração}
\DecValTok{1} \SpecialCharTok{{-}} \DecValTok{1} 
\end{Highlighting}
\end{Shaded}

\begin{verbatim}
## [1] 0
\end{verbatim}

\begin{Shaded}
\begin{Highlighting}[]
\CommentTok{\# Multiplicação}
\DecValTok{2} \SpecialCharTok{*} \DecValTok{2} 
\end{Highlighting}
\end{Shaded}

\begin{verbatim}
## [1] 4
\end{verbatim}

\endColumns
\beginAThirdColumn

\begin{Shaded}
\begin{Highlighting}[]
\CommentTok{\# Divisão}
\DecValTok{4}\SpecialCharTok{/}\DecValTok{2} 
\end{Highlighting}
\end{Shaded}

\begin{verbatim}
## [1] 2
\end{verbatim}

\begin{Shaded}
\begin{Highlighting}[]
\CommentTok{\# Potenciação}
\DecValTok{5}\SpecialCharTok{\^{}}\DecValTok{2} 
\end{Highlighting}
\end{Shaded}

\begin{verbatim}
## [1] 25
\end{verbatim}

\begin{Shaded}
\begin{Highlighting}[]
\CommentTok{\# Radiciação}
\DecValTok{2}\SpecialCharTok{\^{}}\NormalTok{(}\DecValTok{1}\SpecialCharTok{/}\DecValTok{3}\NormalTok{) }
\end{Highlighting}
\end{Shaded}

\begin{verbatim}
## [1] 1.259921
\end{verbatim}

\endColumns
\beginAThirdColumn

\begin{Shaded}
\begin{Highlighting}[]
\CommentTok{\# Resto}
\DecValTok{10} \SpecialCharTok{\%\%} \DecValTok{3}
\end{Highlighting}
\end{Shaded}

\begin{verbatim}
## [1] 1
\end{verbatim}

\begin{Shaded}
\begin{Highlighting}[]
\CommentTok{\# Parte inteira}
\DecValTok{10} \SpecialCharTok{\%/\%} \DecValTok{3} 
\end{Highlighting}
\end{Shaded}

\begin{verbatim}
## [1] 3
\end{verbatim}

\endColumns
\end{frame}

\begin{frame}[fragile]{Funções trigonométricas}
\phantomsection\label{funuxe7uxf5es-trigonomuxe9tricas}
\beginAHalfColumn

\begin{Shaded}
\begin{Highlighting}[]
\CommentTok{\# Seno}
\FunctionTok{sin}\NormalTok{(}\DecValTok{0}\NormalTok{) }
\end{Highlighting}
\end{Shaded}

\begin{verbatim}
## [1] 0
\end{verbatim}

\begin{Shaded}
\begin{Highlighting}[]
\CommentTok{\# Cosseno}
\FunctionTok{cos}\NormalTok{(}\DecValTok{0}\NormalTok{) }
\end{Highlighting}
\end{Shaded}

\begin{verbatim}
## [1] 1
\end{verbatim}

\begin{Shaded}
\begin{Highlighting}[]
\CommentTok{\# Tangente}
\FunctionTok{tan}\NormalTok{(}\DecValTok{0}\NormalTok{) }
\end{Highlighting}
\end{Shaded}

\begin{verbatim}
## [1] 0
\end{verbatim}

\endColumns
\beginAHalfColumn

\begin{Shaded}
\begin{Highlighting}[]
\CommentTok{\# Arco seno}
\FunctionTok{asin}\NormalTok{(}\DecValTok{0}\NormalTok{)}
\end{Highlighting}
\end{Shaded}

\begin{verbatim}
## [1] 0
\end{verbatim}

\begin{Shaded}
\begin{Highlighting}[]
\CommentTok{\# Arco cosseno}
\FunctionTok{acos}\NormalTok{(}\DecValTok{1}\NormalTok{) }
\end{Highlighting}
\end{Shaded}

\begin{verbatim}
## [1] 0
\end{verbatim}

\begin{Shaded}
\begin{Highlighting}[]
\CommentTok{\# Arco tangente}
\FunctionTok{atan}\NormalTok{(}\DecValTok{0}\NormalTok{) }
\end{Highlighting}
\end{Shaded}

\begin{verbatim}
## [1] 0
\end{verbatim}

\endColumns
\end{frame}

\begin{frame}[fragile]{Funções matemáticas especiais}
\phantomsection\label{funuxe7uxf5es-matemuxe1ticas-especiais}
\beginAThirdColumn

\begin{Shaded}
\begin{Highlighting}[]
\CommentTok{\# Exponencial base e}
\FunctionTok{exp}\NormalTok{(}\DecValTok{1}\NormalTok{)}
\end{Highlighting}
\end{Shaded}

\begin{verbatim}
## [1] 2.718282
\end{verbatim}

\begin{Shaded}
\begin{Highlighting}[]
\CommentTok{\# Raiz quadrada}
\FunctionTok{sqrt}\NormalTok{(}\DecValTok{4}\NormalTok{) }
\end{Highlighting}
\end{Shaded}

\begin{verbatim}
## [1] 2
\end{verbatim}

\begin{Shaded}
\begin{Highlighting}[]
\CommentTok{\# Log neperiano}
\FunctionTok{log}\NormalTok{(}\DecValTok{10}\NormalTok{) }
\end{Highlighting}
\end{Shaded}

\begin{verbatim}
## [1] 2.302585
\end{verbatim}

\endColumns
\beginAThirdColumn

\begin{Shaded}
\begin{Highlighting}[]
\CommentTok{\# Log qualquer base}
\FunctionTok{log}\NormalTok{(}\DecValTok{10}\NormalTok{, }\AttributeTok{base =} \DecValTok{5}\NormalTok{) }
\end{Highlighting}
\end{Shaded}

\begin{verbatim}
## [1] 1.430677
\end{verbatim}

\begin{Shaded}
\begin{Highlighting}[]
\CommentTok{\# Fatorial}
\FunctionTok{factorial}\NormalTok{(}\DecValTok{4}\NormalTok{)}
\end{Highlighting}
\end{Shaded}

\begin{verbatim}
## [1] 24
\end{verbatim}

\begin{Shaded}
\begin{Highlighting}[]
\CommentTok{\# Valor absoluto}
\FunctionTok{abs}\NormalTok{(}\SpecialCharTok{{-}}\DecValTok{1}\NormalTok{) }
\end{Highlighting}
\end{Shaded}

\begin{verbatim}
## [1] 1
\end{verbatim}

\endColumns
\beginAThirdColumn

\begin{Shaded}
\begin{Highlighting}[]
\CommentTok{\# Arredondamento para cima}
\FunctionTok{ceiling}\NormalTok{(}\FloatTok{1.2}\NormalTok{)}
\end{Highlighting}
\end{Shaded}

\begin{verbatim}
## [1] 2
\end{verbatim}

\begin{Shaded}
\begin{Highlighting}[]
\CommentTok{\# Arredondamento para baixo}
\FunctionTok{floor}\NormalTok{(}\FloatTok{1.2}\NormalTok{)}
\end{Highlighting}
\end{Shaded}

\begin{verbatim}
## [1] 1
\end{verbatim}

\begin{Shaded}
\begin{Highlighting}[]
\CommentTok{\# Arredondamento}
\FunctionTok{round}\NormalTok{(}\FloatTok{1.2}\NormalTok{, }\AttributeTok{digits =} \DecValTok{0}\NormalTok{)  }
\end{Highlighting}
\end{Shaded}

\begin{verbatim}
## [1] 1
\end{verbatim}

\endColumns
\end{frame}

\begin{frame}[fragile]{Operadores lógicos}
\phantomsection\label{operadores-luxf3gicos}
\beginAThirdColumn

\begin{Shaded}
\begin{Highlighting}[]
\CommentTok{\# São iguais?}
\DecValTok{1} \SpecialCharTok{==} \DecValTok{1} 
\end{Highlighting}
\end{Shaded}

\begin{verbatim}
## [1] TRUE
\end{verbatim}

\begin{Shaded}
\begin{Highlighting}[]
\CommentTok{\# São iguais?}
\DecValTok{1} \SpecialCharTok{==} \DecValTok{2} 
\end{Highlighting}
\end{Shaded}

\begin{verbatim}
## [1] FALSE
\end{verbatim}

\begin{Shaded}
\begin{Highlighting}[]
\CommentTok{\# São diferentes?}
\DecValTok{2} \SpecialCharTok{!=} \DecValTok{2} 
\end{Highlighting}
\end{Shaded}

\begin{verbatim}
## [1] FALSE
\end{verbatim}

\endColumns
\beginAThirdColumn

\begin{Shaded}
\begin{Highlighting}[]
\CommentTok{\# São diferentes?}
\DecValTok{1} \SpecialCharTok{!=} \DecValTok{2} 
\end{Highlighting}
\end{Shaded}

\begin{verbatim}
## [1] TRUE
\end{verbatim}

\begin{Shaded}
\begin{Highlighting}[]
\CommentTok{\# 2 é menor ou igual a 1?}
\DecValTok{2} \SpecialCharTok{\textless{}=} \DecValTok{1}
\end{Highlighting}
\end{Shaded}

\begin{verbatim}
## [1] FALSE
\end{verbatim}

\begin{Shaded}
\begin{Highlighting}[]
\CommentTok{\# 2 é maior ou igual a 1?}
\DecValTok{2} \SpecialCharTok{\textgreater{}=} \DecValTok{1} 
\end{Highlighting}
\end{Shaded}

\begin{verbatim}
## [1] TRUE
\end{verbatim}

\endColumns
\beginAThirdColumn

\begin{Shaded}
\begin{Highlighting}[]
\CommentTok{\# 2 é maior do que 1?}
\DecValTok{2} \SpecialCharTok{\textgreater{}} \DecValTok{1} 
\end{Highlighting}
\end{Shaded}

\begin{verbatim}
## [1] TRUE
\end{verbatim}

\begin{Shaded}
\begin{Highlighting}[]
\CommentTok{\# 2 é menor do que 1?}
\DecValTok{2} \SpecialCharTok{\textless{}} \DecValTok{1} 
\end{Highlighting}
\end{Shaded}

\begin{verbatim}
## [1] FALSE
\end{verbatim}

\endColumns
\end{frame}

\begin{frame}[fragile]{E, OU e NÃO}
\phantomsection\label{e-ou-e-nuxe3o}
\beginAHalfColumn

\begin{itemize}
\item
  Combinação de resultados lógicos.
\item
  1 é menor que 5 \textbf{E} 2 é maior ou igual a 3?
\end{itemize}

\begin{Shaded}
\begin{Highlighting}[]
\NormalTok{(}\DecValTok{1} \SpecialCharTok{\textless{}} \DecValTok{5}\NormalTok{) }\SpecialCharTok{\&}\NormalTok{ (}\DecValTok{2} \SpecialCharTok{\textgreater{}=} \DecValTok{3}\NormalTok{)}
\end{Highlighting}
\end{Shaded}

\begin{verbatim}
## [1] FALSE
\end{verbatim}

\begin{itemize}
\tightlist
\item
  1 é menor que 5 \textbf{OU} 2 é maior ou igual a 3?
\end{itemize}

\begin{Shaded}
\begin{Highlighting}[]
\NormalTok{(}\DecValTok{1} \SpecialCharTok{\textless{}} \DecValTok{5}\NormalTok{) }\SpecialCharTok{|}\NormalTok{ (}\DecValTok{2} \SpecialCharTok{\textgreater{}=} \DecValTok{3}\NormalTok{)}
\end{Highlighting}
\end{Shaded}

\begin{verbatim}
## [1] TRUE
\end{verbatim}

\endColumns
\beginAHalfColumn

\begin{itemize}
\tightlist
\item
  2 é menor que 5? Inverta a resposta lógica.
\end{itemize}

\begin{Shaded}
\begin{Highlighting}[]
\SpecialCharTok{!}\NormalTok{(}\DecValTok{2} \SpecialCharTok{\textless{}} \DecValTok{5}\NormalTok{)}
\end{Highlighting}
\end{Shaded}

\begin{verbatim}
## [1] FALSE
\end{verbatim}

\endColumns
\end{frame}

\begin{frame}[fragile]{Valores especiais}
\phantomsection\label{valores-especiais}
\beginAThirdColumn

\begin{itemize}
\tightlist
\item
  NA: valores ausentes.
\item
  NULL: objetos vazios.
\item
  Inf e -Inf: infinitos.
\item
  NaN: indeterminações.
\end{itemize}

\endColumns
\beginAThirdColumn

\begin{Shaded}
\begin{Highlighting}[]
\DecValTok{1} \SpecialCharTok{+} \ConstantTok{NA}
\end{Highlighting}
\end{Shaded}

\begin{verbatim}
## [1] NA
\end{verbatim}

\begin{Shaded}
\begin{Highlighting}[]
\DecValTok{1} \SpecialCharTok{+} \ConstantTok{NULL}
\end{Highlighting}
\end{Shaded}

\begin{verbatim}
## numeric(0)
\end{verbatim}

\endColumns
\beginAThirdColumn

\begin{Shaded}
\begin{Highlighting}[]
\DecValTok{1}\SpecialCharTok{/}\DecValTok{0}
\end{Highlighting}
\end{Shaded}

\begin{verbatim}
## [1] Inf
\end{verbatim}

\begin{Shaded}
\begin{Highlighting}[]
\DecValTok{0}\SpecialCharTok{/}\DecValTok{0}
\end{Highlighting}
\end{Shaded}

\begin{verbatim}
## [1] NaN
\end{verbatim}

\endColumns
\end{frame}

\section{Variáveis}\label{variuxe1veis}

\begin{frame}[fragile]{Variáveis}
\phantomsection\label{variuxe1veis-1}
\begin{itemize}
\item
  No R podemos usar \textbf{objetos} para atribuir valores que serão
  usados recorrentemente.
\item
  Por exemplo, suponha que estamos trabalhando com o valor \(10\) e que
  este valor será usado várias e várias vezes no código.
\item
  Para poupar algum trabalho, podemos atribuir este valor a um objeto.
  Por exemplo, \texttt{x}:
\end{itemize}

\begin{Shaded}
\begin{Highlighting}[]
\NormalTok{x }\OtherTok{\textless{}{-}} \DecValTok{10}
\end{Highlighting}
\end{Shaded}
\end{frame}

\begin{frame}[fragile]{Variáveis}
\phantomsection\label{variuxe1veis-2}
\begin{itemize}
\item
  Usamos o operador de atribuição \texttt{\textless{}-} (lemos RECEBE)
  para atribuir o valor \(10\) à variável \texttt{x}.
\item
  O sinal de igual (\texttt{=}) também pode ser usado, mas o
  \texttt{\textless{}-} é mais comum e recomendado.
\item
  Note que ao fazer uma atribuição o resultado não é printado no
  terminal.
\item
  Para que o resultado seja printado, digite o nome da variável em uma
  nova linha e execute.
\end{itemize}
\end{frame}

\begin{frame}[fragile]{Variáveis}
\phantomsection\label{variuxe1veis-3}
\begin{itemize}
\item
  Ao fazer uma atribuição, criamos um objeto na nossa área de trabalho.
\item
  Para listar os objetos criados na área de trabalho usamos a função
  \texttt{ls()}.
\item
  Se repetirmos o código de atribuição com outro valor, vamos o
  sobrescrever. Portanto, cuidado!
\item
  Quando há a necessidade de apagar um objeto da área de trabalho usamos
  a função \texttt{rm()}.
\end{itemize}
\end{frame}

\begin{frame}[fragile]{Variáveis}
\phantomsection\label{variuxe1veis-4}
\beginAHalfColumn

\begin{Shaded}
\begin{Highlighting}[]
\CommentTok{\# Atribui o valor 10 à variável x}
\NormalTok{x }\OtherTok{\textless{}{-}} \DecValTok{10}

\CommentTok{\# Printa o valor de x}
\NormalTok{x}
\end{Highlighting}
\end{Shaded}

\begin{verbatim}
## [1] 10
\end{verbatim}

\begin{Shaded}
\begin{Highlighting}[]
\CommentTok{\# Lista as variáveis}
\FunctionTok{ls}\NormalTok{()}
\end{Highlighting}
\end{Shaded}

\begin{verbatim}
## [1] "format_field" "thm"          "x"
\end{verbatim}

\endColumns
\beginAHalfColumn

\begin{Shaded}
\begin{Highlighting}[]
\CommentTok{\# Remove a variável x}
\FunctionTok{rm}\NormalTok{(x)}

\CommentTok{\# Lista as variáveis}
\FunctionTok{ls}\NormalTok{()}
\end{Highlighting}
\end{Shaded}

\begin{verbatim}
## [1] "format_field" "thm"
\end{verbatim}

\endColumns
\end{frame}

\section{Funções básicas}\label{funuxe7uxf5es-buxe1sicas}

\begin{frame}{Funções básicas}
\phantomsection\label{funuxe7uxf5es-buxe1sicas-1}
\begin{itemize}
\item
  Além dos objetos, \textbf{funções} são um elemento importante em R.
\item
  Na prática funções também são objetos.
\item
  O R já vem com diversas funções básicas.
\item
  Algumas já vimos nos operadores matemáticos.
\end{itemize}
\end{frame}

\begin{frame}{Funções básicas}
\phantomsection\label{funuxe7uxf5es-buxe1sicas-2}
\begin{longtable}[]{@{}
  >{\centering\arraybackslash}p{(\linewidth - 2\tabcolsep) * \real{0.2933}}
  >{\centering\arraybackslash}p{(\linewidth - 2\tabcolsep) * \real{0.7067}}@{}}
\toprule\noalign{}
\begin{minipage}[b]{\linewidth}\centering
Função
\end{minipage} & \begin{minipage}[b]{\linewidth}\centering
Tarefa
\end{minipage} \\
\midrule\noalign{}
\endhead
c() & Cria um Vetor \\
setwd() & Muda o Diretório de Trabalho Atual \\
getwd() & Mostra o Diretório de Trabalho \\
dir() & Lista os Arquivos do Diretório de Trabalho Atual \\
sessionInfo() & Mostra algumas informações da sessão \\
install.packages() & Instala um pacote \\
library() & Carrega um pacote \\
require() & Carrega um pacote \\
example() & Mostra exemplos de alguma função \\
print() & Imprime o resultado de uma variável \\
q() & Fecha a Sessão \\
objects() & Exibe objetos criados \\
str() & Mostra a estrutura de um objeto \\
\bottomrule\noalign{}
\end{longtable}
\end{frame}

\section{Funções de ajuda}\label{funuxe7uxf5es-de-ajuda}

\begin{frame}{Funções de ajuda}
\phantomsection\label{funuxe7uxf5es-de-ajuda-1}
\begin{itemize}
\item
  Algo ótimo de se trabalhar em R é a \textbf{documentação interna}.
\item
  Em R cada função e objeto tem a sua própria documentação.
\item
  Em alguns casos, para pacotes que fazem algumas análises especificas,
  temos tutoriais que são chamados de \textbf{vinhetas} (vignettes).
\item
  Existem funções que auxiliam no acesso à documentação.
\end{itemize}
\end{frame}

\begin{frame}[fragile]{Funções de ajuda}
\phantomsection\label{funuxe7uxf5es-de-ajuda-2}
\begin{itemize}
\item
  Para acessar a documentação específica de uma função ou objeto podemos
  usar as funções \texttt{?} ou \texttt{help()}.
\item
  A documentação aparecerá no painel no canto inferior direito do
  RStudio.
\item
  Suponha que você não saiba exatamente o nome da função ou objeto para
  o qual você quer consultar a documentação.
\item
  Neste caso, você pode procurar por funções e objetos por meio de algum
  termo de busca com a função \texttt{help.search()}.
\end{itemize}
\end{frame}

\begin{frame}[fragile]{Funções de ajuda}
\phantomsection\label{funuxe7uxf5es-de-ajuda-3}
\begin{itemize}
\item
  Uma outra forma de obter ajuda é usar a função \texttt{apropos()}.
\item
  Ela vai procurar por objetos no caminho de procura que batem com o
  termo que você está procurando.
\item
  Já as vignettes estão associadas a pacotes específicos.
\item
  Podemos consultar todos os vignettes disponíveis dentro de um pacote
  com a função \texttt{browseVignettes()}.
\item
  O R abrirá uma nova janela em seu browser onde mostrará os títulos dos
  vignettes disponíveis para o pacote solicitado.
\item
  Outra possibilidade é a função \texttt{RSiteSearch()} que fará uma
  busca mais geral procurando pelo termo no site oficial do R
  (r-project.org).
\end{itemize}
\end{frame}

\begin{frame}[fragile]{Funções de ajuda}
\phantomsection\label{funuxe7uxf5es-de-ajuda-4}
\begin{Shaded}
\begin{Highlighting}[]
\CommentTok{\# Solicita a documentação interna do pacote base}
\NormalTok{?base}
\FunctionTok{help}\NormalTok{(base)}

\CommentTok{\# Busca pelo termo \textquotesingle{}linear models\textquotesingle{}}
\FunctionTok{help.search}\NormalTok{(}\StringTok{"linear models"}\NormalTok{)}

\CommentTok{\# Busca funções e objetos pelo nome parcial}
\FunctionTok{apropos}\NormalTok{(}\StringTok{"plot"}\NormalTok{)}

\CommentTok{\# Busca vinhetas}
\FunctionTok{browseVignettes}\NormalTok{(}\StringTok{"grid"}\NormalTok{)}

\CommentTok{\# Busca um termo no site do R}
\FunctionTok{RSiteSearch}\NormalTok{(}\StringTok{"plot"}\NormalTok{)}
\end{Highlighting}
\end{Shaded}
\end{frame}

\begin{frame}{Campos da documentação}
\phantomsection\label{campos-da-documentauxe7uxe3o}
Os campos e seus respectivos conteúdos são os seguintes:

\begin{itemize}
\tightlist
\item
  \textbf{Cabeçalho}: indica o pacote.
\item
  \textbf{Título}: título da função.
\item
  \textbf{Description}: descrição do que o objeto é/faz.
\item
  \textbf{Usage}: como usar ou fazer a chamada.
\item
  \textbf{Arguments}: quais os argumentos formais da função.
\item
  \textbf{Value}: o que a função retorna.
\item
  \textbf{Details}: detalhes adicionais de implementação.
\item
  \textbf{Note}: notas adicionais sobre uso e afins.
\item
  \textbf{See Also}: referências para documentação relacionada.
\item
  \textbf{References}: referências bibliográficas.
\item
  \textbf{Authors}: autores da função.
\item
  \textbf{Examples}: exemplos de uso.
\end{itemize}
\end{frame}

\begin{frame}{Funções de ajuda}
\phantomsection\label{funuxe7uxf5es-de-ajuda-5}
\begin{itemize}
\item
  O R possui uma documentação completa e com diversas opções para
  acesso.
\item
  Não se preocupe em decorar comandos.
\item
  Quando necessário, saiba onde consultar!
\item
  Com o tempo e experiência acabamos nos habituando com diversos
  comandos.
\item
  Mas crie o hábito de acessar documentação interna e também pesquisar
  na web ``como fazer\ldots{} no R''.
\item
  Pesquisas em inglês aumentam a chance de êxito.
\end{itemize}
\end{frame}

\section{Arquivos da linguagem}\label{arquivos-da-linguagem}

\begin{frame}{Arquivos da linguagem}
\phantomsection\label{arquivos-da-linguagem-1}
\begin{itemize}
\item
  Ao usar uma sessão R existem alguns arquivos que são gerados.
\item
  Mencionaremos 2 que costumam ser bastante úteis: \textbf{.Rhistory} e
  \textbf{.RData}.
\item
  \textbf{.Rhistory}: salva todos os comandos executados em uma sessão
  R. Ele é cirado por default ao abrir uma sessão e caso exista um
  .Rhistory no atual diretório de trabalho ele é carregado
  automaticamente.
\item
  \textbf{.RData}: durante uma sessão podemos criar muitos objetos.
  Estes objetos podem ser importantes, de difícil obtenção ou até mesmo
  custosos em termos de tempo e tamanho. O .RData serve para salvar os
  objetos de uma sessão. Ao abrir uma nova sessão e carregar o .RData,
  todos os objetos estarão lá e não precisarão ser gerados novamente.
\end{itemize}
\end{frame}

\begin{frame}[fragile]{Arquivos da linguagem}
\phantomsection\label{arquivos-da-linguagem-2}
\begin{Shaded}
\begin{Highlighting}[]
\CommentTok{\# Salvando o histórico de comandos}
\FunctionTok{savehistory}\NormalTok{(}\AttributeTok{file =} \StringTok{"historico.Rhistory"}\NormalTok{)}

\CommentTok{\# Salvando todas as variáveis criadas na sessão}
\FunctionTok{save.image}\NormalTok{(}\AttributeTok{file =} \StringTok{"variaveis.RData"}\NormalTok{)}
\end{Highlighting}
\end{Shaded}
\end{frame}

\section{Pacotes}\label{pacotes}

\begin{frame}{Pacotes}
\phantomsection\label{pacotes-1}
\begin{itemize}
\item
  No R a principal forma de contribuição da comunidade é por meio de
  pacotes.
\item
  Os pacotes são coleções de funções e/ou conjuntos de dados organizados
  e documentados.
\item
  Um pacote R pode conter código R e também de outras linguagens como C,
  Fortran e C++.
\item
  Alguns pacotes podem depender de bibliotecas do seu sistema
  operacional, as chamadas libs.
\end{itemize}
\end{frame}

\begin{frame}[fragile]{Pacotes}
\phantomsection\label{pacotes-2}
\begin{itemize}
\item
  Existem repositórios oficiais como o CRAN, Biocondutor e MRAN.
\item
  O CRAN o mais famoso e usual.
\item
  A instalação do pacote é feita usando a função
  \texttt{install.packages()}.
\item
  Outra possibilidade é obter arquivos compactados (em geral .tar.gz) e
  fazer uma instalação manual.
\item
  Outra fonte de pacotes são os repositórios de próprios desenvolvedores
  em plataformas de versionamento de código como o Git, GitHub, GitLab,
  entre outros.
\end{itemize}
\end{frame}

\begin{frame}[fragile]{Pacotes}
\phantomsection\label{pacotes-3}
\begin{itemize}
\tightlist
\item
  Por exemplo, podemos instalar o pacote \texttt{ggplot2} pelo próprio
  R.
\end{itemize}

\begin{Shaded}
\begin{Highlighting}[]
\CommentTok{\# instalando o pacote ggplot2 do CRAN}
\FunctionTok{install.packages}\NormalTok{(}\StringTok{"ggplot2"}\NormalTok{)}
\end{Highlighting}
\end{Shaded}

\begin{itemize}
\tightlist
\item
  Para as funções de um pacote poderem ser usadas precisamos carregar o
  pacote com a função \texttt{library()}.
\end{itemize}

\begin{Shaded}
\begin{Highlighting}[]
\FunctionTok{library}\NormalTok{(ggplot2)}
\end{Highlighting}
\end{Shaded}

\begin{itemize}
\tightlist
\item
  Podemos analisar o conteúdo e a documentação de um pacote.
\end{itemize}

\begin{Shaded}
\begin{Highlighting}[]
\FunctionTok{ls}\NormalTok{(}\StringTok{"package:ggplot2"}\NormalTok{) }\CommentTok{\# conteúdo do pacote}
\FunctionTok{help}\NormalTok{(}\AttributeTok{package =} \StringTok{"ggplot2"}\NormalTok{) }\CommentTok{\# documentação do pacote}
\end{Highlighting}
\end{Shaded}
\end{frame}

\begin{frame}{Links}
\phantomsection\label{links}
\beginAHalfColumn

\begin{itemize}
\item
  \href{https://cran.r-project.org/}{\color{blue}{The Comprehensive R Archive Network}}.
\item
  \href{https://posit.co/download/rstudio-desktop/}{\color{blue}{RStudio Desktop}}.
\item
  \href{https://posit.cloud/plans/free}{\color{blue}{Posit Cloud}}.
\end{itemize}

\endColumns
\beginAHalfColumn

\endColumns
\end{frame}

\begin{frame}{Materiais para aprender R}
\phantomsection\label{materiais-para-aprender-r}
\begin{itemize}
\item
  Mayer, FP; Bonat, WH; Zeviani, WM; Krainski, EK; Ribeiro Jr., PJ.
  \href{http://cursos.leg.ufpr.br/ecr/}{\color{blue}{Estatística Computacional com R}}.
  DEST/UFPR, 2018.
\item
  Ribeiro Jr., PJ.
  \href{http://www.leg.ufpr.br/~paulojus/embrapa/Rembrapa/}{\color{blue}{Introdução ao Ambiente Estatístico R}}.
  2011.
\item
  Horton, NJ; Pruim, R; Kaplan, DT.
  \href{http://cran-r.c3sl.ufpr.br/doc/contrib/Horton+Pruim+Kaplan_MOSAIC-StudentGuide.pdf}{\color{blue}{A Student's Guide to R}}.
  2015.
\item
  Maindonald, JH.
  \href{http://cran-r.c3sl.ufpr.br/doc/contrib/usingR.pdf}{\color{blue}{Using R for Data Analysis and Graphics}}.
  2008.
\item
  Paradis, E.
  \href{http://cran-r.c3sl.ufpr.br/doc/contrib/Paradis-rdebuts_en.pdf}{\color{blue}{R for Beginners}}.
  2005.
\end{itemize}
\end{frame}

\begin{frame}{Referências}
\phantomsection\label{referuxeancias}
PENG, Roger D.
\href{https://bookdown.org/rdpeng/rprogdatascience/}{\color{blue}{R programming for data science}}.
Victoria, BC, Canada: Leanpub, 2016.

IHAKA, Ross.
\href{https://www.stat.auckland.ac.nz/~ihaka/downloads/Interface98.pdf}{\color{blue}{R: Past and future history}}.
Computing Science and Statistics, v. 392396, 1998.

Microsoft R Application Network.
\href{https://mran.microsoft.com/documents/what-is-r}{\color{blue}{What is R?}}

The R Project for Statistical Computing.
\href{https://www.r-project.org/about.html}{\color{blue}{What is R?}}
\end{frame}

\end{document}
